\documentclass{article} 
\usepackage{url, graphicx}
\usepackage[margin=1in]{geometry}
\usepackage{textcomp}
\usepackage{algpseudocode}
\usepackage{algorithm}
\usepackage{titling}
\usepackage{amsmath}
\usepackage{amssymb}
\usepackage{amsthm}
\usepackage{verbatim}
\usepackage{listings} % for code highlighting/formatting

\usepackage{color} %defining colors for syntax highlighting
\definecolor{syntaxBlue}{RGB}{42,0.0,255}
\definecolor{syntaxGreen}{RGB}{63,127,95}
\definecolor{syntaxPurple}{RGB}{127,0,85}
\definecolor{syntaxCyan}{RGB}{0,155,155}
\definecolor{syntaxGreyBg}{RGB}{220,220,220}

\lstdefinelanguage{JaTeste} %define the code highlighting/formatting
{
	% list of keywords
	morekeywords={
		func,
		with,
		test,
		if,
		else,
		while,
		for,
		return,
		using,
		import
	},
	sensitive=true, % keywords ARE case-sensitive
	morecomment=[s]{/*}{*/}, % /* and */ delimit comments
	morestring=[b]" % string's MUST be in double quotes
}
\lstset{
	language={JaTeste}, % tell listings package to use the JaTeste language spec
	basicstyle=\small\ttfamily, % Global Code Style
	tabsize=2, % number of spaces indented when discovering a tab 
	columns=fixed, % make all characters equal width
	keepspaces=true, % does not ignore spaces to fit width, convert tabs to spaces
	showstringspaces=false, % lets spaces in strings appear as real spaces
	breaklines=true, % wrap lines if they don't fit
	frame=trbl, % draw a frame at the top, right, left and bottom of the listing
	frameround=tttt, % make the frame round at all four corners
	framesep=4pt, % quarter circle size of the round corners
	numbers=left, % show line numbers at the left
	numberstyle=\tiny\ttfamily, % style of the line numbers
	commentstyle=\color{syntaxGreen},
	keywordstyle=\color{syntaxPurple},
	stringstyle=\color{syntaxBlue},
	emph={int,char,float,struct,string},
	emphstyle=\color{syntaxCyan},
	backgroundcolor=\color{syntaxGreyBg},
}

\title{PLT 4115 LRM: \textbf{JaTest\'{e}}}
\author{
	Andrew Grant\\
	\texttt{amg2215@columbia.edu}
	\and
	Jemma Losh\\
	\texttt{jal2285@columbia.edu}
	\and
	Jared Weiss\\
	\texttt{jbw2140@columbia.edu}
	\and
	Jake Weissman\\
	\texttt{jdw2159@columbia.edu}
}

\date{\today}

\begin{document}

\maketitle

\section{Introduction}

\section{Lexical Conventions}

\subsection{Identifiers}
% Specs on how to name variables, functions, data types, etc.

\subsection{Keywords}
% Just a list of reserved keywords

\subsection{Constants}
% How to define constants such as x = 5

\subsubsection{Integer Constants}
% We should specify all ways you can define an integer


\subsubsection{Character Constants}
% Same for character

\subsubsection{Real Number Constants}
% Do we want to allow for only ints?  If yes, delete this section

\subsubsection{String Constants}
% How to define a string constant

\subsection{Operators}
% Just note they can be used, will be explained more later

\subsection{Separators}
% List all seperators we use including, but not limted to, ( ) [ ] { } ; , . :

\subsection{White Space}

\section{Data Types}

\subsection{Primitives}
% Define primitives and values they can hold

\subsection{Structures}
% I.e. structs

\subsubsection{Defining Structures}

\subsubsection{Initializing Structures}

\subsubsection{Accessing Structure Members}

\subsection{Arrays}

\subsubsection{Defining Arrays}

\subsubsection{Initializing Arrays}

\subsubsection{Accessing Array Elements}

\subsubsection{Multidimensional Arrays}

\subsubsection{Arrays of Structures}
Text

\section{Expressions and Operators}

\subsection{Expressions}

\subsection{Assignment Operators}
% =, +=, -=, etc

\subsection{Incrementing and Decrementing}
% ++, --, etc.

\subsection{Arithmetic Operators}
% +, -, ...

\subsection{Comparison Operators}
% ==, >, <, etc.

\subsection{Logical Operators}
% &&, ||

\subsection{Comma Operator}
% , for sequencing

\subsection{Operator Precedence}

\subsection{Order of Evaluation}
% ++ vs * and such

\section{Statements}

\subsection{Expression Statements}
% 2 + 2;

\subsection{If Statement}
% explain if, else if, else

\subsection{While Statement}

\subsection{For Statement}

\subsection{Code Blocks}
% Code within braces

\subsection{Break and Continue}
% Do we allow for that?

\subsection{Return Statement}

\section{Functions}

\subsection{Function Declarations}

\subsection{Calling Functions}

\subsection{Function Parameters}

\subsection{Main Function}

\subsection{Recursive Functions}

\section{Program Structure and Scope}

\subsection{Program Structure}

\subsection{Scope}

\section{A Sample Program}

\end{document}