\documentclass{article} 
\usepackage{url, graphicx}
\usepackage[margin=1in]{geometry}
\usepackage{textcomp}
\usepackage{algpseudocode}
\usepackage{algorithm}
\usepackage{titling}
\usepackage{amsmath}
\usepackage{amssymb}
\usepackage{amsthm}
\usepackage{verbatim}
\usepackage{listings} % for code highlighting/formatting

\usepackage{color} %defining colors for syntax highlighting
\definecolor{syntaxBlue}{RGB}{42,0.0,255}
\definecolor{syntaxGreen}{RGB}{63,127,95}
\definecolor{syntaxPurple}{RGB}{127,0,85}
\definecolor{syntaxCyan}{RGB}{0,155,155}
\definecolor{syntaxGreyBg}{RGB}{220,220,220}

\lstdefinelanguage{JaTeste} %define the code highlighting/formatting
{
	% list of keywords
	morekeywords={
		func,
		with,
		test,
		if,
		else,
		while,
		for,
		return,
		using,
		import
	},
	sensitive=true, % keywords ARE case-sensitive
	morecomment=[s]{/*}{*/}, % /* and */ delimit comments
	morestring=[b]" % string's MUST be in double quotes
}
\lstset{
	language={JaTeste}, % tell listings package to use the JaTeste language spec
	basicstyle=\small\ttfamily, % Global Code Style
	tabsize=2, % number of spaces indented when discovering a tab 
	columns=fixed, % make all characters equal width
	keepspaces=true, % does not ignore spaces to fit width, convert tabs to spaces
	showstringspaces=false, % lets spaces in strings appear as real spaces
	breaklines=true, % wrap lines if they don't fit
	frame=trbl, % draw a frame at the top, right, left and bottom of the listing
	frameround=tttt, % make the frame round at all four corners
	framesep=4pt, % quarter circle size of the round corners
	numbers=left, % show line numbers at the left
	numberstyle=\tiny\ttfamily, % style of the line numbers
	commentstyle=\color{syntaxGreen},
	keywordstyle=\color{syntaxPurple},
	stringstyle=\color{syntaxBlue},
	emph={int,char,float,struct,string},
	emphstyle=\color{syntaxCyan},
	backgroundcolor=\color{syntaxGreyBg},
}

\title{PLT 4115 LRM: \textbf{JaTest\'{e}}}
\author{
	Andrew Grant\\
	\texttt{amg2215@columbia.edu}
	\and
	Jemma Losh\\
	\texttt{jal2285@columbia.edu}
	\and
	Jared Weiss\\
	\texttt{jbw2140@columbia.edu}
	\and
	Jake Weissman\\
	\texttt{jdw2159@columbia.edu}
}

\date{\today}

\begin{document}

\maketitle

\section{Introduction}
The goal of JaTest\'{e} is to design a language that promotes good coding practices - mainly as it relates to testing.  JaTest\'{e} will require the user to explicitly define test cases for any function that is written in order to compile and execute code.  This will ensure that no code goes untested and will increase the overall quality of programmer code written in our language.  The user will be required to provide some test cases for their code, and the language will also generate some important test cases for their code as well.  JaTest\'{e} is mostly a functional language with a syntax quite similar to C.  The details of our language usage is provided in the rest of the document.

\section{Lexical Conventions}
This chapter will describe how input code will be processed and how tokens will be generated.

\subsection{Identifiers}
% Specs on how to name variables, functions, data types, etc.
Identifiers are used to name a variable, a function, or other types of data.  An identifier can include all letters, digits, and the underscore character.  An identifier must start with either a letter or an underscore - it cannot start with a digit.  Capital letters will be treated differently from lower case letters.

\subsection{Keywords}
% Just a list of reserved keywords
Keywords are a set of words that serve a specific purpose in our language and may not be used by the programmer for any other reason.  The list of keywords the language recognizes and reserves is as follows: 

\texttt{int char float struct if else for while break continue with test using func return}

\subsection{Constants}
% How to define constants such as x = 5

\subsubsection{Integer Constants}
% We should specify all ways you can define an integer

\subsubsection{Character Constants}
% Same for character

\subsubsection{Real Number Constants}
% Do we want to allow for only ints?  If yes, delete this section

\subsubsection{String Constants}
% How to define a string constant

\subsection{Operators}
% Just note they can be used, will be explained more later
Operators are special tokens such as multiply, equals, etc. that are applied to one or two operands.  Their use will be explained further in chapter 4.

\subsection{Separators}
% List all seperators we use including, but not limted to, ( ) [ ] { } ; , . :

\subsection{White Space}

\section{Data Types}

\subsection{Primitives}
% Define primitives and values they can hold

\subsection{Structures}
% I.e. structs

\subsubsection{Defining Structures}

\subsubsection{Initializing Structures}

\subsubsection{Accessing Structure Members}

\subsection{Arrays}

\subsubsection{Defining Arrays}

\subsubsection{Initializing Arrays}

\subsubsection{Accessing Array Elements}

\subsubsection{Multidimensional Arrays}

\subsubsection{Arrays of Structures}
Text

\section{Expressions and Operators}

\subsection{Expressions}
An expression is a collection of one or more operands and zero or more operators that can be evaluated to produce a value.  A function that returns a value can be an operand as part of an expression.  Additionally, parenthesis can be used to group smaller expressions together as part of a larger expression.  A semicolon terminates an expression.  Some examples of expressions include:
\begin{itemize}
\item \texttt{35 - 6};
\item \texttt{foo(42) * 10};
\item \texttt{8 - (9 / (2 + 1))};
\end{itemize}

\subsection{Assignment Operators}
% =, +=, -=, etc
Assignment can be used to assign the value of an expression on the right side to a named variable on the left hand side of the equals operator.  The left hand side can either be a named variable that has already been declared or a named variable that is being declared and initialized in this assignment.  Examples include:

\begin{itemize}
\item \texttt{int x = 5};
\item \texttt{float y};

\texttt{y = 9.9};
\end{itemize}

Additionally, the following operators can also be used for variations of assignment:

\begin{itemize}
\item \texttt{+=} increments the left hand side by the result of the right hand side
\item \texttt{-=} decrements the left hand side by the result of the right hand side
\end{itemize}

\subsection{Incrementing and Decrementing}
% ++, --, etc.
This can be done using the \texttt{++} operator to increment and the \texttt{--} operator to decrement a value.  If the operator is placed before a value it will be incrememnted / decremented first, then it will be evaluated.  If the operator is placed following a value, it will be evaluated with its original value and then incremented / decremented.

\subsection{Arithmetic Operators}
% +, -, ...
\begin{itemize}
\item \texttt{+} can be used for addition
\item \texttt{-} can be used for subtraction (on two operands) and negation (on one operand)
\item \texttt{*} can be used for multiplication
\item \texttt{/} can be used for division
\item \texttt{$\wedge$} can be used for exponents
\item \texttt{$\%$} can be used for modular division
\end{itemize}

\subsection{Comparison Operators}
% ==, >, <, etc.
\begin{itemize}
\item \texttt{==} can be used to evaluate equality
\item \texttt{!=} can be used to evaluate inequality
\item \texttt{>} can be used to evaluate is the left greater than the right
\item \texttt{>=} can be used to evaluate is the left greater than or equal to the right
\item \texttt{<} can be used to evaluate is the left less than the right
\item \texttt{<=} can be used to evaluate is the left less than or equal to the right
\end{itemize}

\subsection{Logical Operators}
% &&, ||
\begin{itemize}
\item \texttt{!} can be used to evaluate the negation of one expression
\item \texttt{$\&\&$} can be used to evaluate logical and
\item \texttt{$\vert\vert$} can be used to evaluate logical or
\end{itemize}

\subsection{Operator Precedence}

\subsection{Order of Evaluation}
% ++ vs * and such

\section{Statements}

\subsection{Expression Statements}
% 2 + 2;

\subsection{If Statement}
% explain if, else if, else

\subsection{While Statement}

\subsection{For Statement}

\subsection{Code Blocks}
% Code within braces

\subsection{Break and Continue}
% Do we allow for that?

\subsection{Return Statement}

\section{Functions}

\subsection{Function Declarations}

\subsection{Calling Functions}

\subsection{Function Parameters}

\subsection{Main Function}

\subsection{Recursive Functions}

\section{Program Structure and Scope}

\subsection{Program Structure}

\subsection{Scope}

\section{A Sample Program}

\end{document}